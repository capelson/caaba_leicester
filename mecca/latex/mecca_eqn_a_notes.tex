% created automatically by xmecca, DO NOT EDIT!

\note{A311MS_a01}{Rate constant derived from k$_3$/k$_2b$ = 6.E-2 in \citet{3437} and k$_3$ = 1.7E5 in \cite{3543}.}

\note{A321MS_a01}{Rate constant is derived from k(NO2p + H2O)/K(NO2p), where K(NO2p) is 1E16 \cite{3913}.}

\note{A41013a_a01}{Branching ratios taken from \citet{3472}}

\note{A41016a_a01}{Branching ratio explaining the \chem{HCOOH}-yield by \citet{3448} who originally assigned it to the channel for the methylic H-abstraction. However, \citet{3448}, differently from \citet{3453}, assumed that the self-reaction of \chem{CH_3O_2} would only produce 2 \chem{CH_3O} radicals and thus \chem{HCHO} + \chem{HO_2}. Instead, the latter reaction has a 0.8 yield of \chem{HOCH_2O_2}, which is a precursor of hydroxymethyl hydroperoxide and thus \chem{HCOOH}.}

\note{A41016b_a01}{The \chem{CH_2OOH} radical has a lifetime of $10^{-9}$ s in the gas phase decomposing to \chem{HCHO} and \chem{OH}. \chem{O_2}-addition in the aqueous-phase seems unlikely. It is hard to imagine how the \chem{HOOHCH_2O_2} radical would decompose into \chem{HCOOH} and \chem{HO_2}.}

\note{A41017a_a01}{$k$(\chem{H_2O_2}+\chem{NO_3}), branching ratio as for \chem{CH_3OOH} + \chem{OH}}

\note{A41017b_a01}{See branch a.}

\note{A41018a_a01}{$k$(\chem{H_2O_2}+\chem{CO_3^-}), branching ratio as for \chem{CH_3OOH} + \chem{OH}}

\note{A41018b_a01}{See branch a.}

\note{A41019a_a01}{Branching ratio as for \chem{CH_3OOH} + \chem{OH}}

\note{A41019b_a01}{\chem{HOCHOOHO_2} is assumed to directly decompose into \chem{CHOOOH} and \chem{HO_2}}

\note{A41020a_a01}{$k$(\chem{H_2O_2}+\chem{NO_3}), branching ratio as for \chem{CH_3OOH} + \chem{OH}}

\note{A41020b_a01}{\chem{HOCHOOHO_2} is assumed to directly decompose into \chem{CHOOOH} and \chem{HO_2}}

\note{A41021_a01}{\chem{HO_2} elimination}

\note{A41022_a01}{$k$(\chem{HO_2}+\chem{HO_2})}

\note{A41023_a01}{$k$(\chem{HO_2}+\chem{O_2^-})}

\note{A41030_a01}{\chem{HO_2} elimination}

\note{A42000a_a01}{\chem{CH_3CHOHO_2} is assumed to directly decompose into \chem{CH_3CHO} + \chem{HO_2}}

\note{A42001a_a01}{\chem{CH_3CHOHO_2} is assumed to directly decompose into \chem{CH_3CHO} + \chem{HO_2}}

\note{A42003_a01}{$k$(\chem{HO_2}+\chem{O_2^-})}

\note{A42004_a01}{$k$(\chem{HO_2}+\chem{HO_2})}

\note{A42005_a01}{$k$ approximated from ($k$(\chem{CH_3OOH}+\chem{OH}) $/k$(\chem{CH_3OH}+\chem{OH}))}

\note{A42006_a01}{\chem{CH_2OHCHOHO_2} is assumed to directly decompose into \chem{HOCH_2CHO} + \chem{HO_2}}

\note{A42007_a01}{\chem{CH_2OHCHOHO_2} is assumed to directly decompose into \chem{HOCH_2CHO} + \chem{HO_2}}

\note{A42010_a01}{$k$ based on \citet{3449}: $k$=$k$(2 \chem{CH_3CH_2(OO)})}

\note{A42014_a01}{\chem{HO_2} elimination}

\note{A42016a_a01}{$k$ assumed to be the same as for \chem{CH_3CHO} + \chem{NO_3}}

\note{A42016b_a01}{See branch a.}

\note{A42017_a01}{$k$(\chem{CH_3CHOHOH}+\chem{O_2^-})}

\note{A42018_a01}{\chem{HO_2} elimination}

\note{A42019_a01}{$k$ based on \citet{3440}}

\note{A42020_a01}{$k$ based on \citet{3440}}

\note{A42022a_a01}{$k$(\chem{CH_3CHOHOH}+\chem{NO_3})}

\note{A42022b_a01}{See branch a.}

\note{A42022c_a01}{See branch a.}

\note{A42023b_a01}{\chem{CHOHOHO_2} directly decomposes into \chem{HCOOH} + \chem{HO_2}}

\note{A42024a_a01}{$k$ based on \cite{70}}

\note{A42024b_a01}{\chem{CHOHOHO_2} directly decomposes into \chem{HCOOH} + \chem{HO_2}}

\note{A42025_a01}{\chem{HO_2} elimination}

\note{A42030b_a01}{\chem{CH_3CHOOHO_2} is assumed to directly decompose into \chem{CH_3CO_2H} and \chem{HO_2}}

\note{A42031b_a01}{\chem{CH_3CHOHO_2} is assumed to directly decompose into \chem{CH_3CHO} + \chem{HO_2}}

\note{A42032_a01}{$k$(\chem{HO_2}+\chem{O_2^-})}

\note{A42033_a01}{$k$(\chem{HO_2}+\chem{HO_2})}

\note{A42034_a01}{$k$(\chem{HO_2}+\chem{HO_2})}

\note{A42035_a01}{$k$(\chem{HO_2}+\chem{O_2^-})}

\note{A42037_a01}{$k$(\chem{HO_2}+\chem{HO_2})}

\note{A42038_a01}{$k$(\chem{HO_2}+\chem{O_2^-})}

\note{A42144a_a01}{$k$ assumed to be the same as \chem{C_2H_5OOH} + \chem{OH} based on \citet{3448}}

\note{A42144b_a01}{See branch a.}

\note{A42146a_a01}{$k$ assumed to be the same as \chem{C_2H_5OOH} + \chem{OH} based on \citet{3448}}

\note{A42146b_a01}{See branch a.}

\note{A42148_a01}{\chem{HO_2} elimination}

\note{A42149_a01}{\chem{HO_2} elimination}

\note{A42150a_a01}{\chem{COOHOO} is not formed but directly dissociates into \chem{CO_2} + \chem{HO_2}. Rate coefficient based on \citet{144}}

\note{A42150b_a01}{See branch a.}

\note{A42151a_a01}{\chem{COOHOO} is not formed but directly dissociates into \chem{CO_2} + \chem{HO_2}. Rate coefficient based on \citet{3461}}

\note{A42151b_a01}{See branch a.}

\note{A42156b_a01}{COOHOO is not formed but directly dissociates into \chem{CO_2} + \chem{HO_2}.}

\note{A42157a_a01}{$k$(\chem{CHOHOHCHOHOH}+\chem{NO_3})}

\note{A42157b_a01}{\chem{COOHOO} is not formed but directly dissociates into \chem{CO_2} + \chem{HO_2}}

\note{A42161_a01}{\chem{HO_2} elimination}

\note{A42162_a01}{\chem{HO_2} elimination}

\note{A42163a_a01}{$k$(\chem{HOCH_2CHO} +\chem{OH})}

\note{A42163b_a01}{See branch a.}

\note{A42164a_a01}{$k$(\chem{HOCH_2CHO}+\chem{NO_3})}

\note{A42164b_a01}{See branch a.}

\note{A42165a_a01}{$k$(\chem{HOCH_2CHOHOH}+\chem{OH})}

\note{A42165b_a01}{See branch a.}

\note{A42165c_a01}{See branch a.}

\note{A42166a_a01}{$k$(\chem{HOCH_2CHOHOH}+\chem{NO_3})}

\note{A42166b_a01}{See branch a.}

\note{A42166c_a01}{See branch a.}

\note{A42167_a01}{pH-dependent}

\note{A42169_a01}{$k=2\times k$(\chem{HOCH_2OH}+\chem{OH})}

\note{A42471_a01}{Assumed to be the same as CH3CO3 + H2O, following \citet{995}}

\note{A42472_a01}{Assumed to be the same as CH3CO3 + H2O, following \citet{995}}

\note{A42473_a01}{Assumed to be the same as CH3CO3 + H2O, following \citet{995}}

\note{A43000a_a01}{Intermidate reaction with \chem{O_2^-} and \chem{CH(OH)_2COCH_2O_2} neglected}

\note{A43001_a01}{\chem{CH(OH)_2COCH_2O_2} neglected}

\note{A43002_a01}{\chem{CO_2} added for mass balance intermediate reactions neglected}

\note{A43004_a01}{\chem{CO_2} added for mass balance intermediate reactions neglected}

\note{A43010a_a01}{\chem{CH_2(OH)COCH_2O_2} was negected with a branching ratio 0.16 added to \chem{CH_3COCHOHO_2}}

\note{A43013_a01}{There is an intermediate reaction with branching ratio 0.87 and 0.13, the minor compound is neglected \citep{3449}}

\note{A43014_a01}{There is an intermediate reaction with branching ratio 0.87 and 0.13, the minor compound is neglected \citep{3483}}

\note{A43015a_a01}{$k$ calculated comparing the rates (\chem{CH_3OH + OH}/\chem{CH_3OOH + OH}) and (\chem{ACETOL + OH}/\chem{HYPERACET + OH})}

\note{A43015b_a01}{$k$ from \chem{CH_3OOH + OH} $\rightarrow$ HCHO.}

\note{A43016_a01}{$k$ taked from the reaction of the hydrated form of \chem{MGLYOX} and \chem{NO_3}}

\note{A43017_a01}{$k$ from \chem{CH_3O_2 + HO_2}}

\note{A43018_a01}{$k$ from \chem{CH_3O_2 + O_2^-}}

\note{A43019a_a01}{$k$ calculated comparing \chem{CH_3OH+OH} / \chem{CH_3OOH+OH} with \chem{IPROPL+OH}}

\note{A43019b_a01}{$k$ calculated comparing \chem{CH3OH+OH} / \chem{CH3OOH+OH} with \chem{ACETOL+OH} / \chem{HYPERCET+OH}}

\note{A43020_a01}{$k$ taken from the reaction of the hydrated form of \chem{MGLYOX} and \chem{NO_3}}

\note{A43021_a01}{$k$ from \chem{CH_3O_2 + HO_2}}

\note{A43022_a01}{$k$ from \chem{CH_3O_2 + O_2^-}}

\note{A43023_a01}{pH-dependent}

\note{A43025_a01}{$k=2\times k$(\chem{HOCH_2OH}+\chem{OH})}

\note{A44010_a01}{$k=2\times k$(\chem{CHOHOHCHOHOH}+\chem{OH})}

\note{A44011_a01}{$k=2\times k$(\chem{CHOHOHCHOHOH}+\chem{OH})}

\note{A44012_a01}{$k=2\times k$(\chem{CHOHOHCHOHOH}+\chem{OH})}

\note{A46000_a01}{Assumed to be the same as for glyoxal}

\note{A46001_a01}{Assumed to be the same as for glyoxal}

\note{A46002_a01}{$k=2\times k$(\chem{CH_3COCHOHOH}+\chem{OH})}

\note{A46003_a01}{Assumed to be the same as for glyoxal}

\note{A46004_a01}{Assumed to be the same as for glyoxal}

\note{A46005_a01}{$k=2\times k$(\chem{CH_3COCHOHOH}+\chem{OH})}

\note{A460MS_a01}{Ryder et al. say that k is the same as for NO2+ + chloride; In reality 6\% 4-nitrophenol, 80\%4-nitrosophenol and 14\% 2-nitrophenol is formed but we assume that only 2-nitrophenol is formed like in the gas phase.}

\note{A61002_a01}{\citet{1008} found an upper limit of \code{6E9} and cite an upper limit from another study of \code{2E9}. Here, we set the rate coefficient to \code{1E9}.}

\note{A63001_a01}{There is also an earlier study by \citet{69} which found a smaller rate coefficient but did not consider the back reaction.}

\note{A64000_a01}{$k$ taken from \chem{H_2O_2}+\chem{Cl_2^-} \citep{1761}.}

\note{A74000_a01}{Assumed to be the same as for \chem{Br_2^-} + \chem{H_2O_2}.}

\note{A76003_a01}{The rate coefficient is defined as backward reaction divided by equilibrium constant.}

\note{A91005_a01}{The rate coefficient for the sum of the paths (leading to either \chem{HSO_5^-} or \chem{SO_4^{2-}}) is from \citet{75}, the ratio 0.28/0.72 is from \citet{111}.}

\note{A91006_a01}{See also: \citep{75,1}. If this reaction produces a lot of \chem{SO_4^-}, it will have an effect. However, we currently assume only the stable \chem{S_2O_8^{2-}} as product. Since \chem{S_2O_8^{2-}} is not treated explicitly in the mechanism, \chem{SO_4^{2-}} is used as a proxy and the second sulfur atom is put into the lumped LSULFUR.}

\note{A92005_a01}{D.\ Sedlak, pers.\ comm.\ (1993).}

\note{A92008_a01}{D.\ Sedlak, pers.\ comm.\ (1993).}

\note{A930MS_a01}{As suggested in Staudt et al., 2019 intermediate product is NO2SO4-, which readily reacts with H2O and produce final products}

\note{A94100_a01}{$2.48\E7 \times 5.5\E{-4}$, considering the hydrated form of HCHO.}

\note{A94102_a01}{$790 \times 5.5\E{-4}$, considering the hydrated form of HCHO.}

\note{A94108a_a01}{$k$(\chem{H_2O_2}+\chem{SO_4^-}), branching ratio as for \chem{CH_3OOH} + \chem{OH}}

\note{A94108b_a01}{See branch a.}

\note{A94200_a01}{$k$(\chem{CH_3OO}+\chem{HSO_3^-})}

\note{A94201_a01}{$k$(\chem{CH_3OO}+\chem{HSO_3^-})}

\note{A94202a_a01}{\chem{CH_3CHOHO_2} is assumed to directly decompose into \chem{CH_3CHO} + \chem{HO_2}}

\note{A94203b_a01}{\chem{CHOHOHO_2} directly decomposes into \chem{HCOOH} + \chem{HO_2}}

\note{A96005_a01}{Assumed to be the same as for \chem{SO_3^{2-}} + \chem{HOCl}.}

\note{A97005_a01}{Assumed to be the same as for \chem{SO_3^{2-}} + \chem{HOBr}.}

\note{A116005_a01}{products assumed}

\note{A119001_a01}{products assumed}

\note{A119002_a01}{products assumed}

\note{A119004_a01}{Assumed. Note that CAPRAM 2.4 lists $k$=4.3E7 from Herrmann Air Pollution Research Report 57 and it also lists $k$= 2.65E7 from Williams PhD 1996 \url{http://lib.leeds.ac.uk/record=b1835184~S5}. \citet{3346} also list $k$=3.56E4 from Waygood EUROTRAC 1992 report.}

\note{A119006_a01}{3E8*6500/(48000+6500)}

\note{A119008_a01}{Assuming that the intermediate \chem{S_2O_6^{2-}} dissociates quickly.}
